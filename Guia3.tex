\documentclass[12pt,a4paper]{article}

% Packages
\usepackage[top=1cm,bottom=2cm,left=1cm,right=1cm]{geometry}%
\usepackage[most]{tcolorbox}
\usepackage{xcolor}

% Graphviz
\usepackage{tikz}
\usepackage{dot2texi}
\usepackage{graphicx}
\usetikzlibrary{shapes.geometric}

% Inputs
% Para matemática.
\usepackage{amsmath}
\usepackage{amssymb}

% Para \demoline
\usepackage{ragged2e}

% Para usar \text
\usepackage{amsmath}
\usepackage{amsfonts}
\usepackage{amssymb}
\usepackage{graphicx}
\usepackage{lmodern}
\usepackage[T1]{fontenc}

\newcommand{\Title}[1]{%
  \raggedright
  \noindent{\textbf{#1}}%
  \justifying
  \vspace{1em}%
}

\newcommand{\demoline}{\vspace{0.5em}}

\newcommand{\Indent}{\hspace*{0.75cm}}
\newcommand{\MiniIndent}{\hspace*{0.325cm}}
\newenvironment{groupIzq}[1]{%
  \begin{list}{}{%
      \setlength{\leftmargin}{#1}%
      \setlength{\topsep}{0pt} % Elimina el espacio superior
      \setlength{\partopsep}{0pt} % Asegura que no haya espacio extra
    }
  \item[]
}{%
  \end{list}
}
\newcommand{\entt}{\ensuremath{\mathbb{Z}}}
\newcommand{\natt}{\ensuremath{\mathbb{N}}}


% Colors
\pagecolor{gray!160}
\color{gray!10}

% Space between \par
\setlength{\parskip}{.35em}

\begin{document}

\Title{Ejercicio 1}
\par Demostrar, usando inducción en la cantidad de aristas, que todo digrafo \ensuremath{D} satisface
\demoline

\begin{math}
  \displaystyle
  \sum_{v \in V(D)} d_{\text{in}}(v)
  = \sum_{v \in V(D)} d_{\text{out}}(v)
  = |E(D)|
\end{math}

\demoline
\par Para poder hacer inducción sobre la cantidad de aristas, primero empezamos definiendo el predicado:
\demoline
\par \ensuremath{P(n)\text{: } ``\text{Todo grafo } G(E, V) \text{ cumple que } \sum_{v \in V(D)} d_{\text{in}}(v) = \sum_{v \in V(D)} d_{\text{out}}(v) = |E| = m"}
\demoline

\par \underline{\textbf{Caso Base (\ensuremath{m = 0}):}}
\par Un digrafo \ensuremath{G=(V, E)} que tiene 0 aristas, cumple simultáneamente:
\begin{groupIzq}{1em}
  \begin{itemize}
    \item \ensuremath{\sum_{v \in V(G)} d_{\text{in}}(v) = 0}
    \item \ensuremath{\sum_{v \in V(G)} d_{\text{out}}(v) = 0}
    \item \ensuremath{|E| = 0}
  \end{itemize}
\end{groupIzq}
\par con lo cual se mantiene la igualdad \ensuremath{\sum_{v \in V(G)} d_{\text{in}}(v) = \sum_{v \in V(G)} d_{\text{out}}(v) = |E| = 0}.
\par Entonces el caso base \textbf{vale}.

\demoline

\par \underline{\textbf{Paso Inductivo:}}
\par Sea \ensuremath{k \in \natt}, y supongamos que \ensuremath{P(k)} vale.
\par Sea \ensuremath{G(V, E)} un digrafo cualquiera, tal que: \ensuremath{|E| = k + 1}.
\par Sea \ensuremath{e = (u, v)}, con \ensuremath{e \in E \land u,v \in V}.
\par Sea \ensuremath{G'=(V, E')}, tal que \ensuremath{E'= E \setminus \{e\} \Rightarrow |E'| = k}.
\demoline
\par Si se cumple que \ensuremath{|E'|=k}, entonces se cumple la HI, con lo cual:
\par \ensuremath{\sum_{v \in V(G')} d_{\text{in}}(v) = \sum_{v \in V(G')} d_{\text{out}}(v) = k}
\demoline
\par ¿Qué pasa si agregamos \ensuremath{e} a \ensuremath{E'}?
\par En primer lugar se cumpliría la igualdad: \ensuremath{G'(V, E' \cup \{e\}) = G(V, E)}.
\par Además, tendría una serie de consecuencias:
\begin{groupIzq}{1em}
  \begin{itemize}
    \item \ensuremath{\sum_{v \in V(G)} d_{\text{in}}(v) = \sum_{v \in V(G')} d_{\text{in}}(v) + 1 = k + 1}
    \item \ensuremath{\sum_{v \in V(G)} d_{\text{out}}(v) = \sum_{v \in V(G')} d_{\text{out}}(v) + 1 = k + 1}
    \item \ensuremath{|E| = |E'| + 1 = k + 1}
  \end{itemize}
\end{groupIzq}
\par Y esto es justamente lo que buscamos!
\par Ahora tenemos que \ensuremath{\sum_{v \in V(G)} d_{\text{in}}(v) = \sum_{v \in V(G)} d_{\text{out}}(v) = |E| = k + 1}.
\par Con lo cual, queda demostrado que:
\par \ensuremath{\forall k \in \natt : P(k) \text{ es verdadero}}.

\demoline
\demoline
\par Fin.

\newpage

\Title{Ejercicio 2}
\par Demostrar, usando la técnica de reducción al absurdo, que todo grafo no trivial tiene al menos dos vértices del mismo grafo. \textbf{Ayuda:} prestar atención a la secuencia ordenada de los grados de los vértices.

\demoline

\par Vamos por partes, ¿Qué es la técnica de reducción al absurdo? (Reductio ad Absurdum)
\par Si se tiene una proposición \ensuremath{P}, se puede suponer que \ensuremath{P} es falsa \ensuremath{(\neg P)} y llegar a una contradicción, en cuyo caso se puede concluir que \ensuremath{P} es verdadera.

\demoline
\par Para eso, definamos la proposición \ensuremath{P}.
\par \ensuremath{P: ``\text{Todo grafo no trivial tiene al menos dos vértices del mismo grado".}}
\par \ensuremath{\neg P: ``\text{Existe al menos un grafo no trivial que tiene todos sus vértices de distinto grado}".}
\par Supongamos que \ensuremath{\neg P} se cumple, o sea que \ensuremath{P} es falsa.
\par Si eso fuera cierto, existiría un grafo \ensuremath{G(V, E)}, con |V| = n, de tal forma que:
\par 
  \ensuremath{
  \left\{ \deg(v) \,\middle|\, v \in V \right\} = \left\{0, \dots, n-1 \right\}
  }
\par Sabemos que estos conjuntos tienen al menos 2 elementos, pues \ensuremath{G} es no trivial.

\demoline

\par Pero en un grafo no puede coexistir un vértice con grado 0 y otro con grado \ensuremath{n-1}
ya que el vértice de grado \ensuremath{n-1}
debería estar conectado con todos los demás, incluido el de grado 0. Contradicción.

\demoline

\par Por la técnica de reducción al absurdo, se concluye que \ensuremath{P} es verdadera.

\demoline
\demoline

\par Fin.

\newpage

\Title{Ejercicio 3}

\par Un grafo orientado es un digrafo \ensuremath{D} tal que al menos uno de \ensuremath{v \rightarrow w} y \ensuremath{w \rightarrow v} no es una arista de \ensuremath{D}, para todo \ensuremath{v, w \in V(D)}.
\par En otras palabras, un grafo orientado se obtiene a partir de un grafo no orientado dando una dirección a cada arista.
\par Demostrar en forma constructiva que para cada \ensuremath{n} existe un único grafo orientado cuyos vértices tienen todos grados de salida distintos.
\par \textbf{Ayuda:} aprovechar el ejercicio anterior y observar que el absurdo no se produce para un único grafo orientado.

\demoline

\par Este problema lo encuentro un poco confuso, ya que el grafo no es único: para cada \ensuremath{n} vamos a más de un digrafo cuyos vértices tienen tienen todos grados de salida distintos.
\par Pero por otro lado considero excelente porque mezcla distintas técnicas de demostración.
\demoline

\par Para demostrar en forma constructiva, primero debemos "armar" el grafo (es decir, exhibirlo).
\par Vamos a tratar de armarlo de la forma más débil posible, es decir: \textbf{ponerle la mínima cantidad de restricciones posibles}.

\demoline

\par Sea \ensuremath{k \in \natt}, \ensuremath{k \geq 2} y sea \ensuremath{V = \{v_1, \dots, v_k\}} un conjunto de vértices.
\par Sea \ensuremath{G(V, E)} un grafo no-dirigido con \ensuremath{|E| = k}, y sea \ensuremath{G'(V, E')}  un digrafo orientado tal que \ensuremath{|E'| = |E|}.
\par \ensuremath{G'} además cumple que: \ensuremath{\{d_\text{out}(v)\text{ / } v \in V\} = \{0, \dots, k-1\}}.

\demoline
\par ¿El mínimo de ese conjunto \textbf{DEBE} ser 0? La respuesta es sí. Veamos por qué eso es así:
\par Sea \ensuremath{z \in \entt, z \geq 0}.
% \par Sea  supongamos que el conjunto empieza con un \ensuremath{n \in \natt \text{ / } n \geq 1}, por ejemplo:
\par Sea \ensuremath{\{d_\text{out}(v)\text{ / } v \in V\} = \{z, \dots, z + (k - 1)\} \Rightarrow \sum_{v \in V}{d_\text{out}(v)} = kz  + \dfrac{k(k - 1)}{2}}
\par Supongamos que \ensuremath{z \geq 1}: \textbf{lleguemos a un absurdo!}.
\par Sabemos que el número máximo de aristas en un digrafo \ensuremath{D(V, E)} no puede exeder \ensuremath{n(n - 1)}. Si \ensuremath{n} es el número de vértices, cada vértice a lo sumo estará conectado a los otros (y no a sí mismo, pues \ensuremath{D} es simple), y como el número máximo de aristas para un determinado vértice es \ensuremath{n - 1}, y como hay \ensuremath{n} vértices: \ensuremath{|E| \leq n(n - 1)}.
\par Entonces:
\par \ensuremath{kz + \dfrac{k(k - 1)}{2} \leq k(k - 1)}
\par \ensuremath{kz \leq k(k - 1) - \dfrac{k(k - 1)}{2}}
\par \ensuremath{z \leq (k - 1) - \dfrac{(k - 1)}{2}}
\par \ensuremath{z \leq \dfrac{(k - 1)}{2}}
\par Habíamos dicho que \ensuremath{k \geq 2} y que \ensuremath{z \geq 1}. Si tomamos \ensuremath{z = 1 \land k = 2} 
\par \ensuremath{1 \leq \dfrac{1}{2}}. Absurdo!
\par Por la ténica de reducción al absurdo, se tiene que \ensuremath{z = 0}.
\demoline
\par Esto nos sugiere que...para cada grafo de \ensuremath{n} vértices, siempre vamos a encontrar un \ensuremath{z} que rompa la cota \ensuremath{n(n-1)}. 
\demoline
\demoline
\demoline
\demoline
\par Pero... ¿Posta que funca?
\par Demostremos que si:
\demoline
\par 
\par Sea \ensuremath{\{d_\text{out}(v)\text{ / } v \in V\} = \{0, \dots, k - 1\} \Rightarrow \sum_{v \in V}{d_\text{out}(v)} = \dfrac{k(k - 1)}{2}}
\par Luego, se efectivamente se verifica que \ensuremath{\dfrac{k(k - 1)}{2} \leq k(k - 1)}, pues \ensuremath{k \geq 2}.
\demoline
\par Ahora ya sabemos \ensuremath{G'} existe! Y además está exhibido.
\par Para completar la demostración, debemos probar que es único, pero realmente... no lo es.
\par Sea \ensuremath{v \in V} con \ensuremath{d_\text{out}(v) > 0 }. Este vértice existe, pues \ensuremath{k \geq 2}, y tomemos una de sus aristas salientes \ensuremath{e \in E'} e invertamos su dirección!
\par Acá se nos abren dos posibilidades, pero ya con la primera sabemos que no es único!
\par Si \ensuremath{k = 2}, tendremos dos tipos de grafos posibles:
\begin{dot2tex}[dot,options=-tmath]
  digraph G {
    color=lightgrey;
    rankdir=LR;
    node [style=filled, fillcolor=black, color=gray, shape=diamond];
    v -> u;
}
\end{dot2tex}
\begin{dot2tex}[dot,options=-tmath]
  digraph G {
    color=lightgrey;
    rankdir=LR;
    node [style=filled, fillcolor=black, color=gray, shape=diamond];
    
    v -> u [dir=back];
}
\end{dot2tex}
\demoline
\par Pero ya vemos que el grafo no es único, vemos que tanto \ensuremath{G'} como \ensuremath{G'^T} sirven.
\demoline
\par Análogamente, para cualquier k, \ensuremath{G'} y \ensuremath{G'^T} funcionan. Pero no solo eso, acá algunas configuraciones que funcionan:
\begin{dot2tex}[dot,options=-tmath]
  digraph G {
    color=lightgrey;
    rankdir=LR;
    node [style=filled, fillcolor=black, color=gray, shape=diamond];
    
    A -> B;
    A -> C;
    B -> C;
}
\end{dot2tex}
\begin{dot2tex}[dot,options=-tmath]
  digraph G {
    color=lightgrey;
    rankdir=LR;
    node [style=filled, fillcolor=black, color=gray, shape=diamond];
    
    A -> B [dir=back];
    A -> C [dir=back];
    B -> C;
}
\end{dot2tex}
\begin{dot2tex}[dot,options=-tmath]
  digraph G {
    color=lightgrey;
    rankdir=LR;
    node [style=filled, fillcolor=black, color=gray, shape=diamond];
    
    A -> B;
    A -> C [dir=back];
    B -> C [dir=back];
}
\end{dot2tex}
\demoline
\par Etcétera.
\par En definitiva, si encontramos un digrafo orientado \ensuremath{G'} que funciona, cualquier digrafo \ensuremath{G''} que sea isomorfismo de \ensuremath{G'}, también sirve.
\demoline
\par Conclusión: No pude probar la unicidad, no desde el punto de vista estructural, al menos.
\end{document}